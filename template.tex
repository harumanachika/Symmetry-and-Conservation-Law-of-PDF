%%% 論文(A)と一般記事用(B)のテンプレート
%%% 必要に応じてどちらかを使ってください.

%%% A. 論文
\documentclass[paper]{jrsj}
\usepackage[T1]{fontenc}
\usepackage{lmodern}
\usepackage{textcomp}
%\usepackage[dvipdfmx]{graphicx}
%\usepackage[dvips]{graphicx}
\usepackage{amsmath}

%% 頁(指定する必要なし)
%\setcounter{page}{1}
%% 通巻頁(指定する必要なし)
%\volpage{101}

%% 論文の種類を記入(学術・技術論文 / 学術論文 / 討論など)
\typeofpaper{学術・技術論文}
%% 発行年(コメントアウトしておく)
%\Year{2018}
%% 巻数(コメントアウトしておく)
%\Vol{36}
%% 号数(コメントアウトしておく)
%\No{1}
%% 邦文題名
\title{}
%% 邦文副題名(必要な場合のみ)
\subtitle{}
%% 英文題名
\etitle{}
%% 英文副題名(必要な場合のみ)
\esubtitle{}
%% 著者のリスト
\authorlist{%
 \authorentry{姓 名}{First Second}{label}
 \authorentry{}{}{}
}
%% 著者の所属
\affiliate[label]{和文所属}{英文所属}
\affiliate[]{}{}

%% 原稿受付日(記述しない,空欄に)
\received{}
%% 本論文は××性(××××分野)で評価されました.
\evaluated{}

\begin{document}
%% 英文要旨
\begin{abstract}
\end{abstract}
%% 英文キーワード
\begin{keywords}
\end{keywords}
\maketitle

%% 本文

%% 謝辞(必要な場合のみ)
\begin{acknowledgements}
\end{acknowledgements}

%% 文献
\begin{thebibliography}{99}
\bibitem{}
\end{thebibliography}

%% 付録(必要な場合のみ)
\appendix 

%% 著者紹介(総合論文,学術・技術論文,解説論文のみ)
\begin{biography}
\profile{m}{名前(Name)}{紹介文}
%\profile{}{}{}
%\profile{}{}{}
\end{biography}

\end{document}


%%%%%%%%%%%%%%%%%%%%%%%%%%%%%%%%%%%%%%%%%%%%%%%%%%%%%%%%%%%%%%%%%%%%%%%%%%%
%% B. 一般記事
\documentclass[article]{jrsj}
\usepackage[T1]{fontenc}
\usepackage{lmodern}
\usepackage{textcomp}
%\usepackage[dvipdfmx]{graphicx}
%\usepackage[dvips]{graphicx}
\usepackage{amsmath}

%% 頁(指定する必要なし)
%\setcounter{page}{1}
%% 通巻頁(指定する必要なし)
%\volpage{101}

%% 一般記事の種類を記入(解説,展望,講座など)
\typeofarticle{解説}% 
%% 発行年(コメントアウトしておく)
%\Year{2018}
%% 巻数(コメントアウトしておく)
%\Vol{36}
%% 号数(コメントアウトしておく)
%\No{1}
%% 邦文題名
\title{}
%% 邦文副題名(必要な場合のみ)
\subtitle{}
%% 英文題名
\etitle{}
%% 英文副題名(必要な場合のみ)
\esubtitle{}
%% 著者のリスト
\authorlist{%
 \authorentry{姓 名}{First Second}{label}
 \authorentry{}{}{}
}
%% 著者の所属
\affiliate[label]{和文所属}{英文所属}{和文住所}{英文住所}
\affiliate[]{}{}{}{}

%% 原稿受付日(記述しない,空欄に)
\received{}
%% 英文キーワード
\keyword{}

\begin{document}
\maketitle

%% 本文

%% 謝辞(必要な場合のみ)
\begin{acknowledgements}
\end{acknowledgements}

%% 文献
\begin{thebibliography}{99}
\bibitem{}
\end{thebibliography}

%% 付録(必要な場合のみ)
\appendix 

%% 著者紹介(総合論文,学術・技術論文,解説論文のみ)
\begin{biography}
\profile{m}{名前(Name)}{紹介文}
%\profile{}{}{}
%\profile{}{}{}
\end{biography}

\end{document}
